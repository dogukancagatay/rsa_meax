\hypertarget{index_intro6}{}\section{Licence}\label{index_intro6}
Dogukan Cagatay 2010 @ Izmir University of Economics

This program is free software; you can redistribute it and/or modify it under the terms of the G\-N\-U General Public License as published by the Free Software Foundation; either version 2 of the License, or (at your option) any later version.

This program is distributed in the hope that it will be useful, but W\-I\-T\-H\-O\-U\-T A\-N\-Y W\-A\-R\-R\-A\-N\-T\-Y; without even the implied warranty of M\-E\-R\-C\-H\-A\-N\-T\-A\-B\-I\-L\-I\-T\-Y or F\-I\-T\-N\-E\-S\-S F\-O\-R A P\-A\-R\-T\-I\-C\-U\-L\-A\-R P\-U\-R\-P\-O\-S\-E. See the G\-N\-U General Public License for more details.

You should have received a copy of the G\-N\-U General Public License along with this program; if not, write to the Free Software Foundation, Inc., 59 Temple Place, Suite 330, Boston, M\-A 02111-\/1307 U\-S\-A

The program basicly implements R\-S\-A key creation, encryption, decryption and authentication. It has 3 modules key creation, message encryption and message authentication.\hypertarget{index_intro}{}\section{Implementation and Libraries}\label{index_intro}
The only externally used library is G\-M\-P (G\-N\-U Mathematical Precision) Library. It is a library which makes it possible to maintain and use big numbers in the code. So most of the mathematical computation, number generation, exponentiation, encryption, decryption, key generation etc. functions are written using the G\-M\-P Library functions. The G\-M\-P library can be obtained from \hyperlink{}{http\-://www.\-gmplib.\-org} . The Secure Hash Algorithm implementation used in the project is taken from another open source project Aescrypt \hyperlink{}{http\-://www.\-aescrypt.\-com} . The hash that S\-H\-A generates, is 256 bits.\hypertarget{index_intro2}{}\section{Compiling The Project}\label{index_intro2}
There are 3 parts in the project that has to be compiled separately. First file is \hyperlink{create__rsa__keys_8c}{create\-\_\-rsa\-\_\-keys.\-c} file. Second file is sender.\-c file and the third is receiver.\-c file. To be able to compile them we need to install the G\-M\-P library from the package manager of your Linux/\-Unix distribution or download and install from its web site \hyperlink{}{http\-://www.\-gmplib.\-com} . To compile the program; 
\begin{DoxyCode}
 gcc -O2 file.c -o file -lm -lgmp
\end{DoxyCode}


After the compilation you can run according to the description below.\hypertarget{index_intro3}{}\section{Running The Programs}\label{index_intro3}
\hypertarget{index_sb1}{}\subsection{Creating Keys}\label{index_sb1}
As we said before the project consists of 3 parts. First part is the key generation, you generate keys with a given user name. User names are required to distinguish the generated keys because each user is creating his own key. Throughout the program the user names are not important, they are just used for the naming of the generated key files. In order to be able to run the \hyperlink{create__rsa__keys_8c}{create\-\_\-rsa\-\_\-keys.\-c} you need 2 arguments. First is the user name, the second is the bit length for the keys. An example to run the program would be like this; 
\begin{DoxyCode}
 ./create_rsa_keys dogukan 1024
\end{DoxyCode}
 It would create 2 files with names \char`\"{}username\-\_\-public\-\_\-key.\-txt\char`\"{} and \char`\"{}username\-\_\-private\-\_\-key.\-txt\char`\"{}. These are the public and private key files that will be used while running the other 2 C program. we need 2 pairs of R\-S\-A keys since we assume we have 2 users. A pair of 1024 bit default sender and receiver key files are put in the related directories. \hypertarget{index_sb2}{}\subsection{Creating Message to send}\label{index_sb2}
The second part is the part we do the encryption using the generated keys. This time we need to give 3 arguments.\-First is the message file in our case it is the id.\-txt in the directory, the second argument is the sender's private key that is going to be used in creation of digital signature. And the third argument is the receivers public key. Example run command should be like; 
\begin{DoxyCode}
 ./send_message input_message_file sender's_private_key receiver's_public_key
\end{DoxyCode}
 After running the program it creates a msg\-\_\-to\-\_\-send file in the same directory it is the file that will be sent to the receiver. \hypertarget{index_sb3}{}\subsection{Authenticating the received message}\label{index_sb3}
The final, third part is the authenticating the received encrypted message. The \hyperlink{authenticate__msg_8c}{authenticate\-\_\-msg.\-c} file also takes 3 arguments. First is the received message file, the second is the receiver's private key file and the third is the sender's public key file. Example run would be like; 
\begin{DoxyCode}
 ./authenticate_msg received_msg_file receiver's_private_key_file sender'
      s_public_key_file
\end{DoxyCode}
 